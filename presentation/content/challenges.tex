% ============================================
\section{Wyzwania}
% ============================================

\begin{frame}[t]{Jakie problemy chcemy rozwiązać?}
\textbf{Plątanina nieznanych pojęć:}
    \pause
\begin{itemize}
    \item nie wiadomo, od czego zacząć
    \pause
    \item wyjaśnienia \emph{ignotum per ignotum}, często zapętlone
    \pause
    \item każdy indywidualny przypadek jest inny --- nie ma jednego wzorca, uniwersalnego wzoru
\end{itemize}
\end{frame}

\begin{frame}[t]{Wybrane technologie i metody analizy danych}
\textbf{Wykorzystujemy nowoczesne narzędzia, które umożliwiają nam zaradzeniu każdemu z tych wyzwań!}
    \pause
\begin{itemize}
    \item Zebrane dane historyczne (inflacja, wskaźnik PKB itp.)
    \pause
    \item Modele uczenia maszynowego przewidujące i klasyfikujące przyrost wynagrodzenia wynikający z progresji kariery
    \pause
    \item Modele językowe LLM wspomagane wyszukiwaniem zasobów sieciowych RAG
\end{itemize}
\end{frame}
