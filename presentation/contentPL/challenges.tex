% ============================================
\section{Wyzwania}
% ============================================

\begin{frame}[t]{Jakie problemy chcemy rozwiązać?}
\textbf{Plątanina nieznanych pojęć:}
    \pause
\begin{itemize}
    \item nie wiadomo, od czego zacząć
    \pause
    \item wyjaśnienia \emph{ignotum per ignotum} (nieznane tłumaczone przez inne nieznane), często zapętlone
    \pause
    \item każdy indywidualny przypadek jest inny --- nie ma jednego schematu, uniwersalnego wzoru
\end{itemize}
\end{frame}

\begin{frame}[t]{Wybrane technologie i metody analizy danych}
\textbf{Wykorzystujemy nowoczesne narzędzia, które umożliwiają nam zaradzeniu każdemu z tych wyzwań!}
    \pause
\begin{itemize}
  \item Dane historyczne (inflacja, \%PKB, płace sektorowe) wprowadzane jako kontekst referencyjny.
  \pause
  \item Modele uczenia maszynowego klasyfikujące zawód/branżę i estymujące ścieżkę wynagrodzeń.
  \pause
  \item Duże modele językowe (LLM) wspomagane RAG – weryfikacja stawek rynkowych i aktualności danych.
\end{itemize}
\end{frame}
