% ============================================
\section{Podsumowanie}
% ============================================

\begin{frame}
  \frametitle{Użycie modeli w aplikacji}
  \textbf{Nasz model przekształca prosty kalkulator składek emerytalnych}
  \textbf{w inteligentnego przewodnika po zależnościach finansowych, który samodzielnie sugeruje}
  \textbf{wartości domyślne kluczowych parametrów.}
  \\
  \pause
  \brand{Zebranie tych danych ręcznie byłoby dla użytkownika czasochłonne i obarczone ryzykiem błędu.}
\end{frame}

% ============================================
\section{Co dalej?}
% ============================================

\begin{frame}
  $\rightarrow$ \textbf{Obecne rozwiązanie (MVP) demonstruje użyteczność wiarygodnych wartości domyślnych.}
  \pause
  \begin{itemize}
    \item Rozwój do w pełni funkcjonalnej aplikacji: stopniowe rozszerzanie parametrów edytowanych przez użytkownika.
          Docelowo każdy parametr będzie możliwy do ustawienia ręcznie, dla bardziej realistycznych prognoz.
    \pause
    \item Ulepszanie modeli, zakresu i jakości danych (źródła publiczne i branżowe),
          a w konsekwencji – wyższa dokładność i możliwość audytu wyników.
  \end{itemize}
\end{frame}